\DeclareMathOperator\diag{diag}
\DeclareMathOperator\QuantileFunction{QF}
\DeclareMathOperator\Min{min}
\DeclareMathOperator\Trace{tr}
\DeclareMathOperator\Spur{sp}

%% Optionally add superscript, if it is not empty
\NewDocumentCommand\Subscripts{ m O{} O{} O{} O{} }{
  %
  % 1 -> object
  % 2 -> default subscript
  % 3 -> subscript
  % 4 -> default superscript
  % 5 -> superscript
  %
  {%
    \providetoggle{shouldbewrapped}%
    \ifstrempty{#2}{}{\ifstrempty{#3}{}{\toggletrue{shouldbewrapped}}}%
    \ifstrempty{#4}{}{\ifstrempty{#5}{}{\toggletrue{shouldbewrapped}}}%
    \iftoggle{shouldbewrapped}{%
      \left(%
      }{}%
      #1%
      \ifstrempty{#2}{}{_{#2}}%
      \ifstrempty{#4}{}{^{#4}}%
      \iftoggle{shouldbewrapped}{%
      \right)%
    }{}%
    \ifstrempty{#3}{}{_{#3}}%
    \ifstrempty{#5}{}{^{#5}}%
  }%
}

%% Optionally add superscript, if it is not empty
\NewDocumentCommand\SubscriptsF{ s m m O{} O{} }{
  %
  % #1 -> default position: top
  % #2 -> object
  % #3 -> floating subscript
  % #4 -> subscript
  % #5 -> superscript
  %
  {%
    \newtoggle{sub}\newtoggle{sup}%
    \newtoggle{full}\newtoggle{empty}%
    \ifstrempty{#4}{}{\toggletrue{sub}}
    \ifstrempty{#5}{}{\toggletrue{sup}}
    \ifboolexpr{togl {sub} and togl {sup}}{\toggletrue{full}}{%
      \ifboolexpr{not ( togl {sub} or togl {sup} )}{\toggletrue{empty}}{}
    }
    \iftoggle{full}{%
      \left(%
    }{}%
    #2%
    \iftoggle{full}{%
      \IfBooleanTF{#1}{^{#3}}{_{#3}}%
      \right)_{#4}^{#5}%
    }{%
      \iftoggle{sub}{_{#4}^{#3}}{}%
      \iftoggle{sup}{^{#5}_{#3}}{}%
      \iftoggle{empty}{%
        \IfBooleanTF{#1}{^{#3}}{_{#3}}%
      }{}%
    }%
  }%
}
%\newcommand\TestSubscripts{%
  %\begin{align}
    %&\Subscripts{A}[te][i][xt][j] \\
    %&\Subscripts{A}[][i][xt][j] \\
    %&\Subscripts{A}[te][][xt][j] \\
    %&\Subscripts{A}[te][i][][j] \\
    %&\Subscripts{A}[te][i][xt][] \\
    %&\Subscripts{A}[][][xt][j] \\
    %&\Subscripts{A}[][i][][j] \\
    %&\Subscripts{A}[][i][xt][] \\
    %&\Subscripts{A}[te][][][j] \\
    %&\Subscripts{A}[te][][xt][] \\
    %&\Subscripts{A}[te][i][][] \\
    %&\Subscripts{A}[te][][][] \\
    %&\Subscripts{A}[][i][][] \\
    %&\Subscripts{A}[][][xt][] \\
    %&\Subscripts{A}[][][][j] 
  %\end{align}
%}
\NewDocumentCommand\MakeIndexedVariable{ s s m m O{} O{} }{%
  % *#1   - wrap into ensuremath
  % *#2   - add xspace
  % {#3}  - command
  % {#4}  - letter
  % [#5]  - 
  % [#6]  - 
  \IfBooleanTF{#1}{%
    \NewDocumentCommand#3{ O{} O{} }{%
      \ensuremath{%
        \Subscripts{#4}[#5][##1][#6][##2]%
      }%
      \IfBooleanTF{#2}{\xspace}{}%
    }%
  }{%
    \NewDocumentCommand#3{ O{} O{} }{%
      \Subscripts{#4}[#5][##1][#6][##2]%
      \IfBooleanTF{#2}{\xspace}{}%
    }%
  }
}
\NewDocumentCommand\MakeIndexedVariableF{ s m m m }{%
  % #1 -> star
  % #2 -> cmdname
  % #3 -> body
  % #4 -> floating index
  \NewDocumentCommand#2{ O{} O{} }{%
    \IfBooleanTF{#1}{%
      \SubscriptsF*{#3}{#4}[##1][##2]%
    }{%
      \SubscriptsF{#3}{#4}[##1][##2]%
    }%
  }%
}

\newcommand*\Constant{\text{const}}
\newcommand*\Minimum{\text{min}}
\newcommand*\Maximum{\text{max}}
\newcommand*\NegInfty{{-\infty}}
\newcommand*\half{\frac{1}{2}}
\newcommand*\PowerHalf{^\half}
\newcommand*\PowerIHalf{^{-\half}}
\newcommand*\inv{\ensuremath{^{-1}}}
\newcommand*\transp{\ensuremath{^T}}
\newcommand*\TranspIfExists[1]{\ifstrequal{#1}{}{}{#1\transp}}
\newcommand*\InvIfExists[1]{\ifstrequal{#1}{}{}{#1\inv}}
\newcommand*\hconjugate{\ensuremath{^\dagger}}
\newcommand*\nohconjugate{\ensuremath{^{\phantom{\dagger}}}}
\newcommand*\cconjugate{\ensuremath{^c}}
\newcommand*\nocconjugate{\ensuremath{^{\phantom{c}}}}
\newcommand*\conjugate{\ensuremath{^*}}
\newcommand*\noconjugate{\ensuremath{^{\phantom{*}}}}
\newcommand*\modulus[1]{\ensuremath{\left\lvert#1\right\rvert}}
\newcommand*\expo[1]{\ensuremath{e^{#1}}}
\newcommand*\squared{\ensuremath{^2}}
\newcommand*\modsquared[1]{\Mod{#1}^2}
\newcommand*\Modsquared[1]{\Mod{#1}^2}
\newcommand*\Mod[1]{\ensuremath{\left\lvert#1\right\rvert}}
\newcommand*\hconjpart{\text{h.c.}}
\newcommand*\Determinant[1]{\left\rvert#1\right\lvert}
\newcommand*\BigO{\mathcal{O}}
\newcommand*\LogDet[1]{\log\Determinant{#1}}
\newcommand*\Mean[1]{\left\langle#1\right\rangle}
\newcommand*\MeanAlt[1]{\overline{#1}}
\if@maxfl@tau
  \newcommand*\TwoPi{\tau}
\else
  \newcommand*\TwoPi{(2\pi)}
\fi

% Math symbols,
% \lagrangian with two unnecessary arguments, sub- and super- scripts.
% \lagrangian[CC]
% \lagrangian[][D]
% needs xparse
\NewDocumentCommand\lagrangian{ O{} O{} }{%
  \ensuremath{%
    \mathcal{L}%
    \ifstrequal{#1}{}{}{_\text{#1}}%
    \ifstrequal{#2}{}{}{^\text{#2}}%
  }%
}
\NewDocumentCommand\Mass{ O{} m }{%
  \ensuremath{%
    M%
    \ifstrequal{#1}{}{}{_\text{#1}}%
    \ifstrequal{#2}{}{}{^\text{#2}}%
  }%
}
\NewDocumentCommand\mass{ O{} m }{%
  \ensuremath{%
    m%
    \ifstrequal{#1}{}{}{_\text{#1}}%
    \ifstrequal{#2}{}{}{^\text{#2}}%
  }%
}
\NewDocumentCommand\ChiSquared{ o }{%
  \ensuremath{%
    \IfValueTF{#1}{\chi^2_\text{#1}}{\chi^2}
  }%
}

%
% Matrix algebra
%
\newcommand*\cvec[1]{\ensuremath{\boldsymbol{#1}}}
\NewDocumentCommand\QuadForm{ m O{} }{% col, mat (optional)
  \ifstrequal{#2}{}{#1^2}{#1^T #2 #1}
}
\NewDocumentCommand\QuadFormI{ m o O{} }{% col, mat (optional), ndim (opt)
  \IfValueTF{#2}{%
    \ifstrequal{#3}{1}{%
      \frac{#1^2}{#2}
    }{%
      \ifstrequal{#2}{}{#1^2}{#1^T \ifstrequal{#2}{I}{}{#2\inv} #1}
    }
  }{%
    #1^2%
  }
}

\NewDocumentCommand\QuadFormDiffI{ m o O{} O{} }{%
  % #1 -> variable,
  % #2 -> covariance (default=none)
  % #3 -> ndim (default=1)
  % #4 -> default value (default=none)
  \QuadFormI{%
    \ifstrequal{#4}{}{#1}{\left(#1-#4\right)}%
  }[#2][#3]%
}

%
% Normal distribution
%
\NewDocumentCommand\NormalDistNorm{ O{} O{1} }{%
  % #1 -> covariance (default=none)
  % #2 -> ndim (default=1)
  \frac{1}{\sqrt{%
      \ifstrequal{#2}{1}{%
        \TwoPi{}#1
      }{%
        \TwoPi^{#2}%
        \ifstrequal{#1}{}{}{\Determinant{#1}}}%
      }%
  }%
}
\NewDocumentCommand\NormalDistExp{ m o O{1} O{} }{%
  % #1 -> variable,
  % #2 -> covariance (default=none)
  % #3 -> ndim (default=1)
  % #4 -> default value (default=none)
  \expo{%
    -\half%
    \QuadFormDiffI{#1}[#2][#3][#4]
  }%
}
\NewDocumentCommand\NormalDistPdf{ m o O{1} O{} }{%
  % #1 -> variable,
  % #2 -> covariance (default=none)
  % #3 -> ndim (default=1)
  % #4 -> default value (default=none)
  \NormalDistNorm[#2][#3]%
  \NormalDistExp{#1}[#2][#3][#4]
}

\NewDocumentCommand\NormalDistCharFunc{ m m m }{%
  % #1 -> variable'
  % #2 -> variable,
  % #3 -> covariance (default=none)
  \expo{i #1\transp #2 - \half #1\transp #3 #1}
}

%\newcommand\TestNormalDistr{%
  %\begin{align}
    %& \NormalDistPdf{x}[V][N][x_0]      & \text{ndim, V, x0}\\
    %& \NormalDistPdf{x}[1][\sigma^2][x_0] & \text{1, sigma, x0}\\
    %& \NormalDistPdf{x}[][M][x_0]       & \text{}\\
    %& \NormalDistPdf{x}[][M]            & \text{}\\
    %& \NormalDistPdf{x}[V][M]           & \text{}
  %\end{align}
%}

%
% Poisson distribution
%
\newcommand*\PoissonDistPdf[2]{%
  % #1 -> n
  % #2 -> mean
  \frac{#2^{#1}e^{-#2}}{#1!}
}

%
% Chi-squared
%
\NewDocumentCommand\ChiSquaredFcn{ m o O{N} O{} o o O{M} O{} }{%
  %
  % Main part:
  % #1 -> x
  % #2 -> V
  % #3 -> N
  % #4 -> x_0
  % Pull terms:
  % #5 -> eta
  % #6 -> V_eta
  % #7 -> M
  % #8 -> eta_0
  \QuadFormDiffI{#1}[#2][#3][#4]%
  \IfValueT{#5}{%
    + \QuadFormDiffI{#5}[#6][#7][#8]%
  }%
}

\NewDocumentCommand\ChiSquareSolution{ m m m O{} o }{%
  %
  % #1 -> data
  % #2 -> covariance
  % #3 -> derivatives matrix
  % #4 -> eta - covariance, if punished
  % #5 -> eta_0
  \left( %
    \ifstrempty{#4}{}{%
    \ifstrequal{#4}{I}{I}{#4\inv} +%
    }%
    \TranspIfExists{#3} #2\inv #3 %
  \right)\inv %
  \TranspIfExists{#3} #2\inv #1%
  \IfValueT{#5}{ + #5 }%
}

\NewDocumentCommand\ChiSquaredCovMatrix{ m m O{} }{%
  %
  % #1 -> covariance
  % #2 -> derivatives matrix
  % #3 -> eta - covariance, if punished
  %
  #1 + #2 #3 #2\transp %
}

\NewDocumentCommand\ChiSquaredMinMatrix{ m m O{} }{%
  %
  % #1 -> covariance
  % #2 -> derivatives matrix
  % #3 -> eta - covariance, if punished
  #1\inv%
  -%
  #1\inv%
  #2%
  \left(%
    \ifstrempty{#3}{}{%
    \ifstrequal{#3}{I}{I}{#3\inv} +%
    }%
    #2\transp%
    #1\inv%
    #2%
  \right)\inv%
  #2\transp%
  #1\inv%
}

\NewDocumentCommand\ChiSquaredMin{ m m m O{} o }{%
  %
  % #1 -> data
  % #2 -> covariance
  % #3 -> derivatives matrix
  % #4 -> eta - covariance, if punished
  #1\transp%
  \left(%
    \ChiSquaredMinMatrix{#2}{#3}[#4]%
  \right)%
  #1%
}

%
% Derivative
%
\newcommand*\Derivative[2]{\frac{d #1}{d #2}}
\newcommand*\DDerivative[3]{\frac{d^2 #1}{d#2\ifstrequal{#3}{}{}{ d#3}}}
\newcommand*\PDerivative[2]{\frac{\partial #1}{\partial #2}}
\NewDocumentCommand\EvaluatedAt{ o m }{%
  \IfValueTF{#1}{%
    \left. #1 \right\rvert_{#2}%
  }{%
    \Bigr|_{#2}%
  }%
}

%
% Integrals
%
\newcommand*\IntSym[1]{\int_{-#1}^{+#1}}
